Guia de Elaboração de Projetos Audiovisuais

\part{Elementos Criativos}

\chapter{Elementos Comuns}

\section{Objetivo}

Informe de maneira clara e sucinta de que se trata o projeto: O título, tipo (ficção, documentário, animação, formato, etc), formato (curta, média ou long-metragem, série, etc), número de episódios (caso seja umasérie), duração prevista (minutagem total de filme ou dois epiśodios), público alvo e janela principal de exibição (sala de cinema, TV paga, TV aberta, VOD, internet, etc).

Diga o que pretende realizar com o projeto: metas e resultados que espera aringir em termos de público. Também é importante estabelescer a compatibilidade entre os objetivos e as estratégias de realização, além dos custos necessários para sua realização.

Deixe claro aindaomo o objetivo do projeto se ajusta às exigências estabelescidasno regulamento dŕea pretendidado edital no qual o projeto será inscrito.

Exemplo:
O objetivo desse projeto é a produção de um filme de longa metragem de focção, intitulado O MENINO NO ESPELHO, baseadona obra homônima de Fernando Sabino, com duração prevista de 80 minutos,ctrçaodo R\$4 lõeesalzado em, cooco oplico ifanto-juvenil.
A janela prioritária de exibição do filmr é o circulo de salas de cinemam com expectativa de 1 milhão de espectadores. As janelas secundárias pretendidas são festivais de cinema, Home Vídeo (DVD e/ou Blu Ray), VOD (Vídeo on demand(, TV fechada e TV aberta, nos mercados brasileiro e internacional.
Projeto: O MENINO NO ESPELHO

\section{Público Alvo}

Quem vai consumir?
Os elementos proncipais que descrevem o público alvo são:
    • Gênero (masculino X feminino)
    • Classe socioeconômica (A, B,C e/ou D)
    • Faixa etária

Pode ser definida de forma abrangente (público infantil, infanto-juvenil, jovem, adulto ou terceira idade) ou de forma mais detalhada (pessoas entr 18 e 40 anos de idade).
Exemplo:

O MENINO NO ESPELHO pretende ter classificação livre, podendo ser considerado um filme para toda a família.. Dentro do público infanto-juvenil, temos como alvo principal crianças de 7 a 11 anos. Quanto à classe econômica, o filme é destinado às classes A, B e C. A partir do momento em que o filme transitar em outras mídias, como a TV aberta, ampliaremos esse perfil, pois o projeto possui poder de comunicação com todas as classes socioeconômicas.
Projeto: O MENINO NO ESPELHO

\section{Logline}
Ttamném chamada de storyline, é a frase que representa a obra audiovisual. É um texto de poucas palavras qye funciona como cartão de visitas do rpojeto. Ela apresenta o esqueleto principal da história em um resumo de seu potencial dramático e é de fácil e rápida compreenção. Deve ser instigante e conter um boa dose de sedução para quem ela será apresentada.

Exemplo:
O modo de vida do interior do Brasil no século XIX, reconstruído a partir da análise da caderneta de encomentdas do Tropeiro João da Cruz.
Projeto: Tropeiros
Produtora: Fazenda Filmes
Autor: Aluísio Salles Jr.

\section{Resumo do Projeto}

Texto curto com cerca de cinco linhas. Deve contar o título da obra, tipo (ficção, animação, documentário ou formato), formato (curta, média ou longa metragem, série, etc.), núm
ero de episódios (caso seja uma série), duração prevista(minutagem total do filme ou dos episódios), janela principal de exibição e versão sintética da sinopse.

Exemplo:
O projeto aqui apresentado prevê a produção de um longa-metragem de ficção, voltaqdo para o público infanto-juvenil, intitulado O MENINO NO ESPELHO, baseado na obra homônima de Fernando Sabino, com duração aproximada de 80 minutos, tendo como janela prioritária o circuito de salas de cinema. O filme narra a história de um garoto que vê sua imagem refletida no espelho tornar-se real.
Projeto: O MENINO NO ESPELHO

\section{Apresentação}

Síntese com o título da obra, tipo (ficção, animação, documentário ou formato), formato (curta, média ou lobga-metragem, série, etc), número de eposódios (caso seja uma série), duração prevista (minutagem total do filme ou dos episódios), público alvo e janela principal de exibição. Descreva de modo sucinto o histórico (origem do projetoe premiações já obtidas) e o objetivo. Acrecente tema, motivaão (o que levou ao desenvolvimento do projeto), premissa (assunto abordado na obra com o impulso dramático que move a história) e relevância.

Se for um projeto que prevê produtos, atividades, formatos em múltiplas plataformas ou desdobramentoda obra principal em outras janelas de exibição, é necessário apresentar o conceito unificador deles.
Se for uma obra de ficção, a apresentação também deve conter gênero dramático (drama, comédia, suspense, etc), tom (drama, comédia, suspense, etc), resumo do enredo com conflito central e previsão de desfecho.

Para projetos de animação, apresente as descrição do universo e suas leis, relação entre personagens, estilo vizual e técnica a ser utilizada.

No caso de projetos de documentário, deve conter a descrição do objeto principal a ser a ser abordado pela obra , a abordagem geral do tema e o estilo documental.

É fundmental ressaltar os objetivos a ser concretixsdos com execução da obra. Isso ajuda Comissão de Seleção ou patronicador quanto as dimensões e ao potencial do projeto. Tanto quanto suas principais qualidades e o potencial do projeto, em termos de impacto cultural, atração de público e trsultado comercial.

Concentre-se em descrever o conteúdo específico do projeto, evitando dissertar sobre as referências visuais, teóticasd ou conceituais que lhe dão suporte. A não ser que o edital espeecífique que essas informações sejam apresentadas na apresentação do projeto.

Em alguns editais a Apresentação pode ser chamada de  Descrição do Projeto, Conceito Geral da Obra ou Proposta de Obra Cinematografica

Projeto: O Menino no Espelho
O Menino no Espelho é um projeto de longa-metragem de ficção, com duração prevista de 80 minutos, um custo estimado de produção estimado em R\$4 milhões, que tem uma janela prioritária de exibição o circuito de salas de cinema.

O Filme é um drama, com tom de comédia, voltado para o público infanto-juvenil, que conta a incrível história de um garoto que vê sua imagem refletida no espelho ganhar vida. Fernando ganha um clone chamado Odnanaref, seu nome ao contrário, que passa a fazer todas as tarefas “chatas” em seu lugar, como enfrentar o valentão da escola ou ficar de castigo trancado dentro de casa.

Essa é a trama central do longa metragem O Menino No Espelho, que tem como premissa básica que é preciso não deixar nunca de ser criança! Com uma história de muita aventura, humor e emoção, o filme trata de valores universais como a infância, a amizade e a descoberta do amor.

No desfecho da trama, Fernando tem a primeira desilusão amorosa.É o começo do fim do meninoe marca o início do processo de amadurecimento. É o menino que se torna homem.

O roteiro é adaptado da obra homônima de Fernando Sabino, um dos mais reconhecidos autores brasileiros. Com mais de 70 edições publicadas, esse livro é adotado até hoje em inúmeras esolas pelo Brasil, habitando o imaginário de várias gerações como forte referência dos tempos de infância.

O roteiro do filme conecta-se com o público de forma emocionante e contundente. A motivação dessa adaptação não é a reprodução de uma obra literária, mas a transposição de um estado de espírito sempre presente no “menino” Fernando. Em um tempo que vivemos com tanta tecnologia e tão pouca imaginação, o Menino No Espelho nos apresenta um mundo de fantasia a ser explorado. Sentimentos e sensações tão humanos tão simples e absolutamente indispensáveis.

Dirigido por Guilherme Fiuza Zenha e com produção da Camisa Listrada, o projeto do filme foi selecionado para importantes encontros de coprodução e laboratórios de roteiro no Brasil e no exteriorcomo o Mainheim-Meetings (Alemanha), Produra au Sud (França)e Laboratório SESC de Roteiros Infato-Juvenis. Além disso, recebeu apoio do Programa Ibemídia na categoria desenvolvimento, venceu o Pitching bi Cibe-Ceará em 2009, o que resultou em apoio de mídia para o seu lançamento, e o pitching da Mostra de Cinema Infantil de Florianópolis em 2010, o que levou o projeto a ser apresentado bi BUFF Film Festival na Suécia em 2011.

Nosso projeto, quando finalizado, pará apelo para distribuição nacional e internacional, pois o filme possui a “infência” como tema central, tendo como o público onfanto-juvenil, grande consumidor de cinema em mtodo mundo. Pouco abandonado no Brasil, sendo e excessão as produções “blockbuster” norte-amenricanas, esse público foi atingido por algumas produções isoladas como por exemplo as séries TAINÁ, MENINO MALUQUINO E O RECENTE O ANO QUE MEUS PAIS SAIRAM DE CASA.

\section{Justificativa}
É um texto de defesa do ropjeto, citando sua relevância e características únicas, que venham a ressaltar sua qualidade e a importância de sua produção.

Elabore a justificativa tendo como objetico responder as seguintes questões.

    • Qual foi sua motivação para propor o projeto?
    • Que circunstâncias favorecem sua execução?
    • Como o projeto atende aos critérios de seleção e qualificação pedidos pelo edital?
    • Qual o diferencial do projeto? ex. Imediatismo, pioneirismo, resgate histórico.
    • Porque ele deve ser realizado?

Lembre-se de que os temas propostos pelo seu projeto nem sempre são de conhecimento dos membros das Comissções de seleção. Por mais qualificados que sejam, nem sempre eles estão contextualizados com o caráter regional, temático e/ou o imediatismo da proposta.

‘Esse texto visa enriquecer a literatura brasileira com um novo personagem. Yom Sawyer, Mogli, Alice, Guliver, Pinocchio e o Pequeno príncipe ganharamum companheiro entre nós: o menino Fernando”.

Essa criação, retirada da orelha do livro “O Menino no Espelho”, representa bem o alcance desse fascinante personagem inspirado nas memórias oessoais da infância do escreito mineiro Fernando Sabino, com mais de 50 livros escritos, como também contos e ensaios, Sabino era um autor com profunda flexibilidade para observar a alma humana e soube como poucos transpor o mundo mágico da infência para a literatura.

O escreitor deixou uma marcante e significativa obra, onde destacamos o romance “Encontro Marcado”, que o projetou nacional e internacionalmente, o romance “O Grande Mentecapto” e “O Homen Nu”, que tornaram-se fikmes de sucesso no brasil. Assim como outros livros do autor, “O Menino no Espelho” é um sucesso editorialque, acreditamos, se repetirá nas telas de cinema,

A adaptação desse livro para o cinema mostra-se como uma grande oportunidade para levarmos um texto de alta beleza e fantasia para o grande público, Temos hoje dentro do mercado áudiovisual uma carênci de filmes brasileiros voltados para o público áudiovisual. E a relevância da realização do longa-metragem “O Menino no Espelho” se dá pelo fato dele possuir qualidades para ocupar parte dessa lacuna. 

Estamos certo de que temos no Brasil excelente matéria preima para realizarmos um cinema inventivo e atrativo para esse público.

Quando finalizadom o longa-metragem “O Menino no espelho” terá apelo para a distribuição nacional e internacional, pois o filme psosui a “infânci”a como tema central,  tendo como alvo o público infanto-juvenil, grande consumidor de cinema em todo o mundo, Pouco abordado no Brasik, sendo a exceção as produções “blockbusters” norte-americanasm esse público foi atingido apenasr apenas   por algumas iniciativas isoladas como por exemplo as séries Taina, Menino Maluquinho e o recente O Ano Que Meus Pais Saíram de Férias

Projeto: O Menino no Espelho

\section{Sinopse}

É uma descrição abreviada, uma apresentação concisa da estrutura narrativa essencial da história (incluindo o conflito, se houver) e dos personagens principais, quando for o caso. 

Uma boa sinopse deve prender a atenção do leitor, fazer com que ele deseje saber mais, fespertar a curiosidade de ler o argumento e o roteiro. Lember-se que os membres da Comissão de Seleção têm uma vistidão de projetos para avaliar em curto prazo. Logo, temos que despertar interesse rapidamente.

A sinopse tem uma estrutura claramente organizada, de construção lógica e cronológica.É um resumo do argumento. O propósito principal dessa simplificação é deixar claros os principais pontos da obra, que usualmente são extensos. Portanto, o leito pode, através da sinopse, adquirir a essncia da história em menos tempo de leitura.
Para projetos de documentário, a sinopse deve conter uma descrição breve do objeto a ser abordado (personagens reais, material de arquivo, manifestações da natureza, etc) e a estratégia de abordagem.

No caso de Reality Show, a sinopse deve apresentar as dinâmicas pré-determinadas de interação entre os personagens reais participantes.
Para programa de variedades ancorado por apresentador, a sinopse deve conter a estrutura essencial do programa com indicação do perfil do apresentador e das situações, dinâmicas, quadros e/ou entrevisstas.
\section{Argumento}

A consreução da idéia de uma obra áudiovisual acontece a partir da elaboração

 do argumento.Ele é peça fundamental de um projeto, um objeto de sedução das Comisspes de Seleção, logo o argumento deve ser arrebatador. A partir dele elabora-re o roteiro.

O ARGUMENTO é um texto corrido que se assemelha a um conto literário. Normalment não tem diálogos nem divisão de sequências, mas, por exemplo, se os diálogos forem importantes para esclarecer os traçõs típicos do gênero da obra, podem ser incluídos.

Para projetos de ficção ou animação, o argumento deve conter o desenvolvimento dramatúrgico, um resumo da trama da obra audiovisual, localizando-a no tempo e no espaço, abordardando o tom e a relação entre os personagens. Se for pertinemte, destaque os grandes blocos narrativos da história, o jogo dos pontos de vista, eventuais intervenções não-dramática e a relação ou importância dessas intervenções com a trama.

A visão sobre o No caso de projetos de documentátio, o atgumento deve apresentar sobre o tema, localizando-o no tempo e no espaço. Deve, ainda, relacionar o objeto principal a ser abordado, a estratégia de abordagem e a sugesrão de estrutura.

Para um Reality Show, o argumento deve conter a descreição dos arranjos originais de criação tẽcnica, artiística e econômica necessários à realização da obra audiovisual.








